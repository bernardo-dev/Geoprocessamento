% !TEX TS-program = pdflatex
% !TEX encoding = UTF-8 Unicode

% This is a simple template for a LaTeX document using the "article" class.
% See "book", "report", "letter" for other types of document.

\documentclass[11pt]{article} % use larger type; default would be 10pt

\usepackage[brazil]{babel}
\usepackage[utf8]{inputenc} % set input encoding (not needed with XeLaTeX)

%%% Examples of Article customizations
% These packages are optional, depending whether you want the features they provide.
% See the LaTeX Companion or other references for full information.

%%% PAGE DIMENSIONS
\usepackage{geometry} % to change the page dimensions
\geometry{a4paper} % or letterpaper (US) or a5paper or....
% \geometry{margin=2in} % for example, change the margins to 2 inches all round
% \geometry{landscape} % set up the page for landscape
%   read geometry.pdf for detailed page layout information

\usepackage{graphicx} % support the \includegraphics command and options

% \usepackage[parfill]{parskip} % Activate to begin paragraphs with an empty line rather than an indent

%%% PACKAGES
\usepackage{booktabs} % for much better looking tables
\usepackage{array} % for better arrays (eg matrices) in maths
\usepackage{paralist} % very flexible & customisable lists (eg. enumerate/itemize, etc.)
\usepackage{verbatim} % adds environment for commenting out blocks of text & for better verbatim
\usepackage{subfig} % make it possible to include more than one captioned figure/table in a single float
% These packages are all incorporated in the memoir class to one degree or another...

%%% HEADERS & FOOTERS
\usepackage{fancyhdr} % This should be set AFTER setting up the page geometry
\pagestyle{fancy} % options: empty , plain , fancy
\renewcommand{\headrulewidth}{0pt} % customise the layout...
\lhead{}\chead{}\rhead{}
\lfoot{}\cfoot{\thepage}\rfoot{}

%%% SECTION TITLE APPEARANCE
\usepackage{sectsty}
\allsectionsfont{\sffamily\mdseries\upshape} % (See the fntguide.pdf for font help)
% (This matches ConTeXt defaults)

%%% ToC (table of contents) APPEARANCE
\usepackage[nottoc,notlof,notlot]{tocbibind} % Put the bibliography in the ToC
\usepackage[backend=biber, style=authoryear, bibstyle=numeric, sorting=none]{biblatex}
\addbibresource{referencias.bib}
\usepackage[titles,subfigure]{tocloft} % Alter the style of the Table of Contents
\renewcommand{\cftsecfont}{\rmfamily\mdseries\upshape}
\renewcommand{\cftsecpagefont}{\rmfamily\mdseries\upshape} % No bold!

%%% END Article customizations

%%% The "real" document content comes below...

% \title{Mapeamento de Resiliência e Acessibilidade Dinâmica do Transporte Público em Ouro Preto-MG}

\title{Modelagem Geoespacial da Mobilidade de Ouro Preto usando o Sistema Hexagonal H3}

\author{Bernardo Sant' Anna Costa}
\date{Novembro 21, 2025} % Activate to display a given date or no date (if empty),
         % otherwise the current date is printed 

\begin{document}
\maketitle

\section{Introdução}
% Cenário geral do tema. Mobilidade urbana em cidades históricas e acidentadas.
A gestão da mobilidade urbana em cidades com topografias acidentadas e preservações históricas apresentam desafios que vão além do planejamento de tráfego, especialmente com o aumento da população e a demanda por transportes sustentáveis e eficientes. No caso a literatura recente aponta que a resiliência das redes de transporte (que é a capacidade da rede de manter a funcionalidade mesmo com falhas ou desastres) é uma questão central para o planejamento urbano \autocite{azolin2019transporte}. 

% O problema específico de Ouro Preto.
O município de Ouro Preto é um caso onde o turismo, a geografia acidentada e a preservação do patrimônio histórico impõem desafios únicos à mobilidade urbana.
Conforme apontado por \autocite{gomes2024caos}, a configuração da malha viária impõem a existência de pontos críticos que quando bloquados podem isolar áreas inteiras da cidade. Por exemplo a região do Morro da Forca e da Rua Padre Rolim estão em áreas de alto risco geológico. Eventos recentes de deslizamentos de terra representam riscos à segurança física e impedimento ao acesso a serviços públicos essenciais como saúde e educação.

% O que falta expandir nas abordagens/soluções atuais.
Mesmo sendo um problema relevante e atual, as abordagens tradicionais de análise espacial muitas vezes não são capazes de capturar a granularidade necessária da mobilidade urbana de Ouro Preto. Modelos baseados em zoneamento de bairros podem generalizar demais os padrões de deslocamento, e uma análise puramente topológica deixa de integrar dados heterogêneos como por exemplo mapas de risco e uso do solo.

% A proposta deste artigo.
Por isso a análise de dados da mobilidade urbana é uma ferramenta importante para a gestão pública e o planejamento urbano. Permitindo a compreensão dos padrões de deslocamento, encontrar gargalos no sistema de transporte e propor melhorias que o tornem resilientes. Por isso, este trabalho propõe a utilização do sistema de indexação hexagonal H3 \autocite{h3geoHome}. O objetivo é simular cenários de bloqueio na malha viária de Ouro Preto e avaliar o impacto na acessebilidade dos serviços públicos essenciais tal qual o da rede de transporte público. Com essa abordagem de células hexagonais espera-se prover os gestores públicos uma ferramenta com integra dados heterogêneos e granular da mobilidade urbana, permitindo a gestão de riscos e a melhoria da resiliência do sistema de transporte público.

\section{Justificativa}

% A urgência do tema.
A análise de resiliência e acessebilidade da mobilidade em cidades com topografia complexa e com restrições viárias históricas, como Ouro Preto, exige uma metodologia que vá além das abordagens tradicionais. A escolha de usar o Sistema de indexação Hexagonal H3 \autocite{h3geoHome} é fundamentada com base nas vantagens técnicas que superam as limitações das abordagens tradicionais.

% Justificativa técnica e metodológica.
O H3 é um sistema de grade global que divide o superfície da Terra em células hexagonais. Com essa base geométrica específica o H3 é superior para a agregação dos dados em comparação com células quadradas ou triangulares tradicionais \autocite{h3geoAggregation}.

\begin{figure}[h!]
    \centering
    \includegraphics[width=1\textwidth]{Imagens/Vizinhos das diferentes formas geométricas.png}
    \caption{Distâncias de um triângulo aos seus vizinhos (esquerda), de um quadrado aos seus vizinhos (centro) e de um hexágono aos seus vizinhos (direita).}
    \label{fig:vizinhanca}
\end{figure}

A esolha da resolução 9, 10 ou 11 são importantes pois afetarão a granularidade da célula e dos dados que estão representando \autocite{h3geoTablesCell}. Então para indexar os pontos críticos da malha viária de Ouro Pretom, sendo o bloqueio de um hexagono capaz de representar o bloqueio de um trecho específico como o Morro da Forca ou um determinado ponto da Rua Padre Rolim.

% Justificativa prática.
A necessidade de simulação da perda de acessebilidade se encaixa na capacidade do H3 de modelar o fluxo de movimento \autocite{h3geoFlowModelling}. No caso deste trabalho isso permite que:
\begin{itemize}
    \item Seja associado pesos para cada aresta direcionada. Pode ser o tempo da viagem.
    \item Modelar dinâmicamente. O bloqueio de hexágono com risco geológico pode ser simmulado e ser feito a análise da propagação dos efeitos na cidade de Ouro Preto.
    \item Recalcular novas rotas na nova configuração da malha viária. Permitindo o planejamento de rotas alternativas em casos de bloqueios.
\end{itemize}

\section{Objetivo}
O objetivo deste trabalho é desenvolver e executar uma simulação baseada no sistema de indexação hexagonal H3 para modelar a resiliência da malha viária de Ouro Preto, simulando o impacto de obstruções causadas por riscos geológicos na acessibilidade urbana.

O primeiro passo é estruturar base territorial de Ouro Preto em células hexagonais H3, integrando dados geoespaciais relevantes como a malha viária, pontos de ônibus, as rotas do transporte público, pontos críticos de risco geológico e a localização dos serviços públicos essenciais.

Em seguida, será implementada uma simulação que permita o bloqueio dinâmico de hexágonos representando áreas de risco, avaliando o impacto na acessibilidade dos serviços públicos e na eficiência do transporte público.

Depois buscar rotas alternativas para mitigar os efeitos dos bloqueios, analisando a resiliência da rede de transporte urbano.
Com as novas rotas e a nova configuração da malha viária, as simulações serão executadas novamente até que se obtenha uma configuração resiliente. Sugerindo novas rotas para o transporte público e identificando áreas que necessitam de intervenções urbanas para melhorar a resiliência.


\section{Revisão Bibliográfica}

\subsection{Conceitos Básicos}

\subsection{Trabalhos Correlatos}

\section{Materiais e Métodos}

\section{Resultados Alcançados}

\section{Considerações Finais}

\subsection{Limitações e Trabalhos Futuros}

\newpage
\printbibliography

\end{document}
