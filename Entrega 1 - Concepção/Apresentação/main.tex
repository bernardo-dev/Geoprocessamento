% Inbuilt themes in beamer
\documentclass{beamer}

\usepackage[brazil]{babel}
\usepackage[utf8]{inputenc}

% Theme choice:
\usetheme{Berlin}

\title{Modelagem Geoespacial da Mobilidade de Ouro Preto Usando o Sistema
Hexagonal H3}
\author{Bernardo Sant' Anna Costa}
\institute{Universidade Federal de Ouro Preto}
\date{\today}

\begin{document}

\frame{\titlepage}

\begin{frame}
\frametitle{Roteiro}
\tableofcontents
\end{frame}

\section{Introdução}
\begin{frame}
\frametitle{Contextualização}
\begin{itemize}
    \item \textbf{Cenário}: Ouro Preto - Topografia acidentada e preservação histórica.
    \item \textbf{Desafio}: Gestão da mobilidade urbana em cidades históricas.
    \item \textbf{Fatores Críticos}: Turismo, aumento populacional e riscos geológicos.
    \item \textbf{Problema}: A configuração da malha viária impõe gargalos que, quando bloqueados, isolam regiões inteiras.
\end{itemize}
\end{frame}

\begin{frame}
\frametitle{Problema de Pesquisa}
\begin{itemize}
    \item Vulnerabilidade a deslizamentos (ex: Morro da Forca, Rua Padre Rolim).
    \item Risco de interrupção do acesso a serviços essenciais (saúde, educação).
    \item \textbf{Lacuna}: Abordagens tradicionais (zoneamento ou topologia simples) falham em capturar a granularidade e a heterogeneidade dos dados (risco + uso do solo).
\end{itemize}
\end{frame}

\section{Justificativa}
\begin{frame}
\frametitle{Justificativa da Escolha da Ferramenta (H3)}
\begin{itemize}
    \item \textbf{Definição}: Sistema global de indexação espacial em grade hexagonal.
    \item \textbf{Vantagens sobre outras geometrias}:
    \begin{itemize}
        \item \textbf{Equidistância}: Vizinhos têm a mesma distância do centro (superior a quadrados/triângulos para análise de vizinhança).
        \item \textbf{Modelagem de Fluxo}: Ideal para simular movimento e propagação em redes.
    \end{itemize}
    \item \textbf{Granularidade}: Resoluções finas (9, 10, 11) permitem representar trechos de rua e pontos de risco com precisão.
\end{itemize}
\end{frame}

\section{Objetivo}
\begin{frame}
\frametitle{Objetivos}
\begin{block}{Objetivo Geral}
Desenvolver e executar uma simulação baseada no sistema hexagonal H3 para modelar a resiliência da malha viária de Ouro Preto frente a riscos geológicos.
\end{block}
\begin{block}{Objetivos Específicos}
\begin{itemize}
    \item Estruturar base territorial de Ouro Preto em células H3.
    \item Simular bloqueios dinâmicos em áreas de risco.
    \item Avaliar impacto na acessibilidade de serviços essenciais.
    \item Propor rotas alternativas para mitigar efeitos.
\end{itemize}
\end{block}
\end{frame}

\section{Revisão Bibliográfica}
\begin{frame}
\frametitle{Conceitos Básicos: Resiliência}
\begin{block}{Definição Robusta}
A resiliência de redes de transporte é a capacidade do sistema de \textbf{manter sua funcionalidade} e níveis de serviço, ou recuperar-se eficientemente, mesmo diante de falhas estruturais, desastres naturais ou perturbações severas.
\end{block}
\end{frame}

\begin{frame}
\frametitle{Trabalhos Correlatos}
\begin{itemize}
    \item Estudos sobre resiliência em redes de transporte urbano.
    \item Aplicações de grades hexagonais (H3) em análise espacial.
    \item Modelagem de riscos geológicos em cidades históricas.
\end{itemize}
\end{frame}

\section{Materiais e Métodos}
\begin{frame}
\frametitle{Metodologia Proposta}
A abordagem metodológica estrutura-se nas seguintes etapas:
\begin{enumerate}
    \item \textbf{Estruturação Territorial}: Discretização do espaço usando células H3.
    \item \textbf{Integração de Dados Heterogêneos}: Fusão de dados de malha viária, transporte público, altimetria e mapas de risco.
    \item \textbf{Modelagem de Rede}: Criação de grafo direcionado com pesos (tempo/custo).
    \item \textbf{Simulação de Cenários}: Remoção dinâmica de células (nós).
    \item \textbf{Análise de Acessibilidade}: Recálculo de rotas e métricas.
\end{enumerate}
\end{frame}

\begin{frame}
\frametitle{Aplicação: Ouro Preto}
\begin{itemize}
    \item \textbf{Dados}: Integração da malha de Ouro Preto com pontos de risco geológico.
    \item \textbf{Simulação}: Bloqueio virtual de hexágonos em áreas críticas (ex: deslizamento no Morro da Forca).
    \item \textbf{Avaliação}: Medir impacto no transporte público e acesso a serviços.
\end{itemize}
\end{frame}

\section{Resultados Alcançados}
\begin{frame}
\frametitle{Resultados Preliminares}
\begin{itemize}
    \item Definição da arquitetura da simulação baseada em H3.
    \item Levantamento e pré-processamento dos dados geoespaciais de Ouro Preto.
    \item Identificação dos pontos críticos de risco geológico na malha viária.
    \item Estruturação do modelo de dados para integração das camadas.
\end{itemize}
\end{frame}

\section{Considerações Finais}
\begin{frame}
\frametitle{Conclusão}
\begin{itemize}
    \item A abordagem H3 oferece granularidade adequada para o problema.
    \item A ferramenta permitirá aos gestores públicos antecipar cenários de crise.
    \item Contribuição para a segurança e eficiência da mobilidade em Ouro Preto.
\end{itemize}
\end{frame}

\begin{frame}
\frametitle{Limitações e Trabalhos Futuros}
\begin{block}{Limitações}
\begin{itemize}
    \item Disponibilidade de dados em tempo real.
    \item Complexidade computacional da simulação em alta resolução.
\end{itemize}
\end{block}
\begin{block}{Trabalhos Futuros}
\begin{itemize}
    \item Implementação de interface visual para gestores.
    \item Expansão para outros tipos de riscos (enchentes, eventos).
\end{itemize}
\end{block}
\end{frame}

\end{document}